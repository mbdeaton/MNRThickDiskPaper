\documentclass[aps,prd,twocolumn,superscriptaddress,groupedaddress]{revtex4}
% Intended to be built with pdflatex, using the REVTEX 4 package.
% Template taken from FermiLab APS Jouranl template.

% packages
\usepackage{amsmath}              % for \text, align, split ...
\usepackage{bm}                   % for bold greek letters in math mode
\usepackage{braket}               % for \bra \ket
\usepackage{graphicx}             % for \includegraphics
\usepackage{hyperref}             % for \url and \href
\usepackage[bbgreekl]{mathbbol}   % for bb greek letters like \bbrho
\usepackage{amssymb}              % for \gtrsim ... (load after mathbbol to
%                                   use amssymb's serif'd Latin \mathbb)

% commands/initialization
\hyphenation{ALPGEN} % avoid incorrect hyphenation
\hyphenation{EVTGEN} % ''
\hyphenation{PYTHIA} % ''

% *******************************************************************************
\begin{document}

\widetext
\leftline{Compiled \today}
\leftline{To be submitted to PRD}

\title{The Matter Neutrino Resonance around Thick Relativisitic Disks}

\author{M.\ Brett Deaton}
\affiliation{Joint Institute for Nuclear Astrophysics,
  Michigan State University, East Lansing, MI 48824, USA}
\affiliation{Department of Physics,
  North Carolina State University, Raleigh, NC 27695, USA}
\email{mbdeaton@ncsu.edu}

\author{Yonglin Zhu}
\affiliation{Department of Physics,
  North Carolina State University, Raleigh, NC 27695, USA}

\author{Evan O'Connor}
\affiliation{Department of Physics,
  North Carolina State University, Raleigh, NC 27695, USA}
 
\author{G.\ C.\ McLaughlin}
\affiliation{Department of Physics,
  North Carolina State University, Raleigh, NC 27695, USA}
\affiliation{Joint Institute for Nuclear Astrophysics,
  Michigan State University, East Lansing, MI 48824, USA}

% *******************************************************************************

\begin{abstract}
  This is our abstract.
\end{abstract}

\maketitle

Some neutrino emitting disk models are described in \cite{fouc2015-bhns_m1} and
\cite{fouc2016-nsns}, and baryonic wind models are described in
\cite{metz2014-red_or_blue} and \cite{fern2015-outflows_bh_spin}.

\section{Overview of MNR in Postmerger disks}
\section{Parameterized disk model}
\section{Formulation using ray tracing}
\section{Results}

% *******************************************************************************
\appendix

\section{Definitions}

This is a reference appendix; it won't be in the final paper.

Vectors are typeset like $\bm{\Omega}$,
and matrices are typeset like $\mathbb{H}$.
We use naturalized units in which $\hbar=c=1$.

\subsection*{Trajectories}
\label{ssc:trajectories}
Each trajectory is labeled by a pair of vectors
giving some position on the trajectory, $\bm{x}$,
and the normalized momentum at that position,
$\bm{\Omega}\equiv\bm{p}/\varepsilon$.
A family of trajectories shares the same $\bm{x}$; for example,
the family of emission trajectories is ($\bm{x}_e,\bm{\Omega}$).
%and the family of observation trajectories is ($\bm{x}_o,\bm{\Omega}$).

Since $\bm{\Omega}$ is normalized, it really only has two degrees of freedom,
which we can make explicit by defining it with respect to spherical polar
angles
\begin{equation}
  \label{eq:angle_def}
  \bm{\Omega} := (\sin\alpha\cos\beta,\sin\alpha\sin\beta,\cos\alpha).
\end{equation}

Each trajectory is parameterized by proper distance, $s$, increasing in
the direction of $\bm{\Omega}$. We choose $s=0$ at $\bm{x}_e$.
%and $s=s_o$ at $\bm{x}_o$.

A neutrino trajectory in flat spacetime obeys the geodesic equations
\begin{align}
  \frac{d\bm{x}}{ds} = \bm{\Omega}, \qquad {\rm and} \qquad
  \frac{d\bm{\Omega}}{ds} = 0. \nonumber
\end{align}
A solution may be written with respect to the emission point,
\begin{equation}
  \label{eq:trajectory_x_solution_from_xe}
  \bm{x}(s) = \bm{x}_e + \int_{0}^{s}ds'\bm{\Omega}.
\end{equation}
%or the observation point,
%\begin{equation}
%  \label{eq:trajectory_x_solution_from_xo}
%  \bm{x}(s) = \bm{x}_o + \int_{s_o}^{s}ds'\bm{\Omega}.
%\end{equation}

\subsection*{Density matrices and flavor evolution}
The neutrino field at some point in phase space ($\bm{x};\bm{p}$)
may be specified by the density matrix, $\bbrho(\bm{x};\bm{p})$.
Given some fluid configuration, we can get $\bbrho$ as the solution to
the quantum kinetic equations for neutrinos.
We can't solve those for a general fluid configuration,
but as an approximation, we can start by solving the Boltzmann equation for each
flavor separately, giving us the neutrino distribution functions
$f_{0,\nu_\alpha}(\bm{x};\bm{p})$. Then we compute
\begin{equation}
  \label{eq:def_unoscillated_rho}
  \bbrho_0(\bm{x};\bm{p}) =
  \sum\limits_{\alpha}f_{0,\nu_\alpha}(\bm{x};\bm{p})
  \ket{\nu_\alpha} \bra{\nu_\alpha},
\end{equation}
where $\ket{\nu_\alpha}$ are the neutrino flavor eigenstates, and
\begin{equation}
  \label{eq:density_evolution}
  \bbrho(\bm{x};\bm{p}) =
  \mathbb{S}(\bm{x};\bm{p})\,
  \bbrho_0(\bm{x};\bm{p})\,
  \mathbb{S}^\dagger(\bm{x};\bm{p}),
\end{equation}
where $\mathbb{S}$ is the neutrino flavor evolution matrix.
The antineutrino density matrix obeys
$\bar{\bbrho} = \bar{\mathbb{S}}^*\,\bar{\bbrho}_0\,\bar{\mathbb{S}}^T$,
with $\bar{\mathbb{S}}$ the antineutrino flavor evolution matrix.

This approximation works as long as incoherent scattering is
`insignificant' along the neutrino trajectory.
(I have not figured out precisely what `insignificant' means here.)
At every incoherent scattering
event producing a neutrino of definite flavor $\nu_\beta$,
$\bbrho$ resets to $\ket{\nu_\beta}\bra{\nu_\beta}$, and
$\mathbb{S}$ resets to the identity matrix.
So in th formalism that follows,
the emission point, $\bm{x}_e$, marks the point of the
last incoherent scattering event along our test trajectory.

The neutrino flavor evolution matrix is found by integrating
\begin{equation}
  \label{eq:S_evolution}
  i\frac{d}{ds} \mathbb{S}(\bm{x};\bm{p}) =
  \mathbb{H}(\bm{x};\bm{p})\,
  \mathbb{S}(\bm{x};\bm{p})
\end{equation}
along a given test trajectory ($\bm{x}_e;\bm{\Omega}$).
The antineutrino evolution matrix obeys the same equation with
$i\rightarrow-i$, and $\mathbb{H}\rightarrow\mathbb{H}^*$.
We define the Hamiltonian below.

Note that along this test trajectory $\bm{x}$ obeys 
Eqn.~\ref{eq:trajectory_x_solution_from_xe}.
We can also write our momentum in the form of an energy and an angle
with $\bm{p}=\varepsilon\bm{\Omega}$.
From here on we will use this notation in our arguments:
\begin{equation}
  \mathbb{H}(\bm{x};\bm{p}) \rightarrow
  \mathbb{H}(s;\varepsilon,\bm{\Omega}). \nonumber
\end{equation}

\subsection*{Flavor evolution Hamiltonian}
We analyze $\mathbb{H}$ in terms of its vacuum, matter, and neutrino
contributions,
\begin{equation}
  \mathbb{H}(s;\varepsilon,\bm{\Omega}) \equiv
  \mathbb{H}_V(\varepsilon) +
  \mathbb{H}_e(s) +
  \mathbb{H}_{\nu\nu}(s;\bm{\Omega}).
\end{equation}
Below we define these terms in the flavor basis.

The vacuum term is
\begin{equation}
  \mathbb{H}_V(\varepsilon) \equiv \frac{1}{4\varepsilon}
  \mathbb{U}
  \mathbb{M}
  \mathbb{U}^\dagger,
\end{equation}
with
\begin{equation}
  \mathbb{M} :\equiv
  \left(
  \begin{matrix}
    -\Delta m^2_{21} & 0 & 0 \\
    0 & \Delta m^2_{21} & 0 \\
    0 & 0 & \Delta m^2_{31}+\Delta m^2_{32}
  \end{matrix}
  \right),
\end{equation}
and $\mathbb{U}$ the unitary mixing matrix, which is defined for example
in \cite{giun2007-nu_phys_and_astro}.
The squared-mass splittings are
\begin{align}
  \Delta m^2_{21} &= 7.59\times10^{-5}\,{\rm eV}^2, \nonumber\\
  \Delta m^2_{32} &= 2.43\times10^{-3}\,{\rm eV}^2, \,{\rm and} \nonumber \\
  \Delta m^2_{31} &= \Delta m^2_{21}+\Delta m^2_{32}. \nonumber
\end{align}

The matter term is
\begin{equation}
  \mathbb{H}_e(s) :\equiv
  \left(
  \begin{matrix}
    V_e(s) & 0 & 0 \\
    0 & 0 & 0 \\
    0 & 0 & 0
  \end{matrix}
  \right),
\end{equation}
where
\begin{align}
  V_e(s) & \equiv \sqrt{2} G_F (n_{e^-}-n_{e^+}) \nonumber \\
  \label{eq:matter_potential}
  &= \sqrt{2} G_F \frac{1}{m_N}\varrho(s)Y_e(s) 
\end{align}
with $G_F$ the Fermi constant,
$n_{e^-}-n_{e^+}$  the net electron number density,
$\varrho$ the rest mass density,
$Y_e$ the electron fraction, and
$m_N$ the mass per nucleon.

The neutrino self-interaction term is
\begin{multline}
  \label{eq:H_nunu_full_form_1}
  \mathbb{H}_{\nu\nu}(s;\bm{\Omega}) \equiv
  \sqrt{2} G_F
  \int \frac{d^3p'}{(2\pi)^3}
  (1-\Omega\cdot\Omega') \\
  \times
  \Big[\bbrho(s;\bm{p}')-\bar{\bbrho}(s;\bm{p}')\Big],
\end{multline}
where $\bm{p}'=\varepsilon' \bm{\Omega}'$ is an ambient neutrino
momentum. We can rewrite Eqn.~\ref{eq:H_nunu_full_form_1} using
Eqn.~\ref{eq:density_evolution} and
$d^3p'=\varepsilon'^2 d\varepsilon' d\Omega'$:
\begin{multline}
  \label{eq:H_nunu_full_form_2}
  \mathbb{H}_{\nu\nu}(s;\bm{\Omega}) =
  \frac{\sqrt{2} G_F}{(2\pi)^3}
  \oint d\Omega' \int d\varepsilon' \,\varepsilon'^2
  (1-\Omega\cdot\Omega') \\
  \times \Big[
    \mathbb{S}(s;\varepsilon',\bm{\Omega}')
    \bbrho_0(s;\varepsilon',\bm{\Omega}')
    \mathbb{S}^\dagger(s;\varepsilon',\bm{\Omega}') \\
    -\bar{\mathbb{S}}^*(s;\varepsilon',\bm{\Omega}')
    \bar{\bbrho}_0(s;\varepsilon',\bm{\Omega}')
    \bar{\mathbb{S}}^T(s;\varepsilon',\bm{\Omega}')
    \Big].
\end{multline}
To compute this at a single point on our trajectory, $\bm{x}(s)$,
we need $\mathbb{S}(s)$ and $\bar{\mathbb{S}}(s)$ at all
energies $\varepsilon'$ and angles $\bm{\Omega}'$. In the formalism
we have developed thusfar, we can only get these by integrating
Eqn.~\ref{eq:S_evolution} along all trajectories in the family
sharing the point $\bm{x}(s)$; our problem is heavily recursive.

\subsection*{Single-angle approximation}
However, if we assume that every trajectory ($\bm{x}(s);\bm{\Omega}$)
yields the same flavor evolution, then we can make the substitution
\begin{equation}
 \mathbb{S}(s;\varepsilon',\bm{\Omega}') \rightarrow
 \mathbb{S}(s;\varepsilon',\bm{\Omega}), \nonumber
\end{equation}
and similarly for $\bar{\mathbb{S}}$, giving us
\begin{multline}
  \label{eq:H_nunu_single_angle}
  \mathbb{H}_{\nu\nu}(s;\bm{\Omega}) =
  \frac{\sqrt{2} G_F}{(2\pi)^3}
  \oint d\Omega' \int d\varepsilon' \,\varepsilon'^2
  (1-\Omega\cdot\Omega') \\
  \times \Big[
    \mathbb{S}(s;\varepsilon',\bm{\Omega})
    \bbrho_0(s;\varepsilon',\bm{\Omega}')
    \mathbb{S}^\dagger(s;\varepsilon',\bm{\Omega}) \\
    -\bar{\mathbb{S}}^*(s;\varepsilon',\bm{\Omega})
    \bar{\bbrho}_0(s;\varepsilon',\bm{\Omega}')
    \bar{\mathbb{S}}^T(s;\varepsilon',\bm{\Omega})
    \Big].
\end{multline}

\subsection*{Self-interaction potentials}
Having applied the single-angle approximation, we can now factor
Eqn.~\ref{eq:H_nunu_single_angle} and perform the angle integration
before we compute $\mathbb{S}$ and $\bar{\mathbb{S}}$:
\begin{multline}
  \label{eq:H_nunu_factored}
  \mathbb{H}_{\nu\nu}(s;\bm{\Omega}) = \\
  \int d\varepsilon' \,
  \mathbb{S}(s;\varepsilon',\bm{\Omega})
  \bigg[
    \frac{1}{\Delta\varepsilon'}\mathbb{V}_0(s;\varepsilon',\bm{\Omega})
    \bigg]
  \mathbb{S}^\dagger(s;\varepsilon',\bm{\Omega}) \\
  -\int d\varepsilon' \,
  \bar{\mathbb{S}}^*(s;\varepsilon',\bm{\Omega})
  \bigg[
    \frac{1}{\Delta\varepsilon'} \bar{\mathbb{V}}_0(s;\varepsilon',\bm{\Omega})
    \bigg]
  \bar{\mathbb{S}}^T(s;\varepsilon',\bm{\Omega}).
\end{multline}
To simplify our equations we have defined
\begin{multline}
  \label{eq:nu_potential_matrix_1}
  \mathbb{V}_0(s;\varepsilon',\bm{\Omega}) \equiv \\
  \frac{\sqrt{2} G_F}{(2\pi)^3}\varepsilon'^2\Delta\varepsilon'
    \oint d\Omega' 
    (1-\Omega\cdot\Omega')
    \bbrho_0(s;\varepsilon',\bm{\Omega}').
\end{multline}
And $\bar{\mathbb{V}}_0$ is defined similarly with respect to $\bar{\bbrho}_0$.
Note the new energy factor $\Delta\varepsilon'$ in our definition
in Eqn.~\ref{eq:nu_potential_matrix_1}:
this term gives $\mathbb{V}_0$ units of energy.
We will find it simplifies things when we sum $\mathbb{V}_0$ over energy bins.

Since $\bbrho_0$ and $\bar{\bbrho}_0$ are diagonal, we further
define unoscillated neutrino potentials by
\begin{multline}
  \label{eq:nu_potential_matrix_2}
  \mathbb{V}_0(s;\varepsilon',\bm{\Omega}) :\equiv \\
  \left(
  \begin{matrix}
    V_{0,\nu_e}(s;\varepsilon',\bm{\Omega}) & 0 & 0 \\
    0 & V_{0,\nu_x}(s;\varepsilon',\bm{\Omega}) & 0 \\
    0 & 0 & V_{0,\nu_x}(s;\varepsilon',\bm{\Omega})
  \end{matrix}
  \right),
\end{multline}
and $\bar{\mathbb{V}}_0$ is the same matrix with
$V_{0,\nu_\alpha} \rightarrow V_{0,\bar{\nu}_\alpha}$
(but of course $V_{0,\nu_{\{\mu,\tau\}}}=V_{0,\bar{\nu}_{\{\mu,\tau\}}}$).

For completeness, the neutrino potentials in
Eqn.~\ref{eq:nu_potential_matrix_2} are defined
\begin{multline}
  \label{eq:nu_potential}
  V_{0,\nu_\alpha}(s;\varepsilon',\bm{\Omega}) \equiv \\
  \frac{\sqrt{2} G_F}{(2\pi)^3}\varepsilon'^2 \Delta\varepsilon'
  \oint d\Omega'
  (1-\Omega\cdot\Omega')
  f_{0,\nu_\alpha}(s;\varepsilon',\bm{\Omega}'),
\end{multline}
This follows from Eqn.~\ref{eq:def_unoscillated_rho}.

\subsection*{Discretizing the integration}
Now we can break our evolution equation, Eqn.~\ref{eq:S_evolution},
into $N_\varepsilon$ coupled equations.
We will integrate this system along our test trajectory
($\bm{x}_e,\bm{\Omega}$).
At each step along the trajectory,
we combine the results from each energy equation in the self-interaction term,
$\mathbb{H}_{\nu\nu}(s;\bm{\Omega})$, using Eqn.~\ref{eq:H_nunu_factored}.
That integral becomes a discrete sum over the $N_\varepsilon$ energy bins:
\begin{multline}
  \label{eq:H_nunu_sum}
  \mathbb{H}_{\nu\nu}(s;\bm{\Omega}) = \\
  \sum\limits_{i=1}^{N_\varepsilon}
  \mathbb{S}(s;\varepsilon'_i,\bm{\Omega})
  \big[
    \mathbb{V}_0(s;\varepsilon'_i,\bm{\Omega})
    \big]
  \mathbb{S}^\dagger(s;\varepsilon'_i,\bm{\Omega}) \\
  - \sum\limits_{i=1}^{N_\varepsilon}
  \bar{\mathbb{S}}^*(s;\varepsilon'_i,\bm{\Omega})
  \big[
    \bar{\mathbb{V}}_0(s;\varepsilon'_i,\bm{\Omega})
    \big]
  \bar{\mathbb{S}}^T(s;\varepsilon'_i,\bm{\Omega}).
\end{multline}
Note that the energy bin size $\Delta\varepsilon'$ in
Eqn.~\ref{eq:H_nunu_factored} has canceled with the energy bin size arising from
the discretization of the integral,
$\int d\varepsilon' \rightarrow \sum \Delta\varepsilon'$.

We compute $\mathbb{V}_0$ and $\bar{\mathbb{V}}_0$ before-hand by sampling
the neutrino distribution functions at a grid of points in $\varepsilon$,
$\cos\alpha$, and $\beta$.
(These angles are defined in Eqn.~\ref{eq:angle_def}.)
Eqn.~\ref{eq:nu_potential} discretizes to
\begin{multline}
  \label{eq:nu_potential_sum}
  V_{0,\nu_\alpha}(s;\varepsilon'_i,\cos\alpha,\beta) \equiv \\
  \frac{\sqrt{2} G_F}{(2\pi)^3}\varepsilon_i^{'2} \Delta\varepsilon'
  \sum\limits_j^{N_\alpha} \sum\limits_k^{N_\beta}
  \Delta(\cos\alpha') \Delta\beta'\\
  \times \big(1-\cos\alpha\cos\alpha_j'-\sin\alpha\sin\alpha_j'\cos(\beta-\beta_k')\big)\\
  \times f_{0,\nu_\alpha}(s;\varepsilon'_i,\cos\alpha_j',\beta_k').
\end{multline}

Our grid of samples is composed of
$N_\varepsilon \times N_\alpha \times N_\beta$ points on cell centers.
The points are
\begin{align}
  \varepsilon'_i \qquad&=\qquad \varepsilon'_{\rm min}
  +\left(i-\frac{1}{2}\right)\Delta\varepsilon' \nonumber \\
  \cos\alpha'_j  \qquad&=\qquad (\cos\alpha')_{\rm min}
  +\left(j-\frac{1}{2}\right)\Delta(\cos\alpha')' \nonumber \\
  \beta'_k       \qquad&=\qquad \beta'_{\rm min}
  +\left(k-\frac{1}{2}\right)\Delta\beta', \nonumber
\end{align}
with indices beginning at 1, and all bin sizes defined like
$\Delta a=(a_{\rm max}-a_{\rm min})/N_a$.

\subsection*{Oscillated self-interaction potential}
The Matter-Neutrino Resonance may occur along the test trajectory
where $\mathbb{H}_e$ and $\mathbb{H}_{\nu\nu}$ are the same
magnitude and differ by sign. If $\mathbb{S}$ and $\bar{\mathbb{S}}$ didn't
evolve from identity, the self-interaction Hamiltonian would take the form
\begin{equation}
  \mathbb{H}_{\nu\nu,0}(s;\bm{\Omega}) :=
  V_{\nu\nu,0}(s;\bm{\Omega})
  \left(
  \begin{matrix}
    1 & 0 & 0 \\
    0 & 0 & 0 \\
    0 & 0 & 0
  \end{matrix}
  \right),
\end{equation}
where
\begin{multline}
  V_{\nu\nu,0}(s;\bm{\Omega}) \equiv \\
  \sum\limits_i
  \big(V_{0,\nu_e}(s;\varepsilon'_i,\bm{\Omega})
  -V_{0,\bar{\nu}_e}(s;\varepsilon'_i,\bm{\Omega})\big).
\end{multline}
This is a nice quantity to compare to the matter potential
in Eqn.~\ref{eq:matter_potential}: an MNR resonance may occur at
$V_{\nu\nu,0}(s;\bm{\Omega})+V_e(s)\approx0$.

However, as flavor evolution does take place, $\mathbb{H}_{\nu\nu}$
develops a complex structure.
We define the oscillated self-interaction potential as
\begin{equation}
  V_{\nu\nu}(s;\bm{\Omega}) \equiv
  \big[\mathbb{H}_{\nu\nu}(s;\bm{\Omega})\big]_{ee}
  - \frac{1}{3}Tr\,\mathbb{H}_{\nu\nu}(s;\bm{\Omega}).
\end{equation}
As an MNR evolves, the flavor evolution seems to force this quantity to obey
$V_{\nu\nu}(s;\bm{\Omega})+V_e(s)\approx0$
(as long as possible).

\bibliography{references}

\end{document}
